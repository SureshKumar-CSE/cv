\usepackage[left=0.70in,right=0.70in,top=0.50in,bottom=0.45in,nohead,includefoot]{geometry}
\usepackage{hyperref}
\usepackage{fancyhdr}
\usepackage{titlesec}
\usepackage{titling}
\usepackage{fontspec}

%%%%%%%%%%%
% CV INFO %
%%%%%%%%%%%

\newcommand{\CVAuthor}{Emily Herbert}
\newcommand{\CVTitle}{CV}
\newcommand{\CVEmail}{emilyherbert@cs.umass.edu}
\newcommand{\CVPhone}{(260) 241-6872}
\newcommand{\CVNote}{\begin{footnotesize}Compiled on {\today}, available at \href{\CVWebpageCV}{\CVWebpageCV}.\end{footnotesize}}
\newcommand{\CVWebpage}{emilyherbert.com}
\newcommand{\CVWebpageCV}{github.com/eherbert/cv.git}
\newcommand{\CVSpotifyLink}{spotify:user:1271178627}

%%%%%%%%%%%%%%
% PAGE STYLE %
%%%%%%%%%%%%%%

\pagestyle{fancy}
\fancyhf{}
\renewcommand{\headrulewidth}{0pt}
\rfoot{\CVNote}
%\rfoot{\thepage}

\setlength{\parindent}{0in}
\setlength{\parskip}{0in}

\widowpenalties 1 10000
\clubpenalties 1 10000
\raggedbottom
\sloppy

\renewcommand{\thefootnote}{\fnsymbol{footnote}}

%%%%%%%%%%%%%%%%
% DATE FORMATS %
%%%%%%%%%%%%%%%%

% Abbreviations for months.
\newcommand{\LongMonth}[1]{%
\ifcase#1\relax
\or January%
\or February%
\or March%
\or April%
\or May%
\or June%
\or July%
\or August%
\or September%
\or October%
\or November%
\or December%
\fi}
\newcommand{\ShortMonth}[1]{%
\ifcase#1\relax
\or Jan%
\or Feb%
\or Mar%
\or Apr%
\or May%
\or Jun%
\or Jul%
\or Aug%
\or Sep%
\or Oct%
\or Nov%
\or Dec%
\fi}

% Datestamp format (default): {yyyy}{MM}{dd} ---> MMM yyyy (e.g., Jan 2008).
\newcommand{\DatestampYMD}[3]{\mbox{\ShortMonth{#2} #1}}
\newcommand{\DatestampYM}[2]{\mbox{\ShortMonth{#2} #1}}
\newcommand{\DatestampY}[1]{#1}
\DeclareOption{MMMyyyy}{} % default

% Datestamp format: {yyyy}{MM}{dd} ---> dd MMM yyyy (e.g., 15 Jan 2008).
\DeclareOption{ddMMMyyyy}{%
\renewcommand{\DatestampYMD}[3]{\mbox{#3 \ShortMonth{#2} #1}}
\renewcommand{\DatestampYM}[2]{\mbox{\ShortMonth{#2} #1}}
\renewcommand{\DatestampY}[1]{#1}}

% Datestamp format: {yyyy}{MM}{dd} ---> MMMM yyyy (e.g., January 2008).
\DeclareOption{MMMMyyyy}{%
\renewcommand{\DatestampYMD}[3]{\mbox{\LongMonth{#2} #1}}
\renewcommand{\DatestampYM}[2]{\mbox{\LongMonth{#2} #1}}
\renewcommand{\DatestampY}[1]{#1}}

% Datestamp format: {yyyy}{MM}{dd} ---> dd MMMM yyyy (e.g., 15 January 2008).
\DeclareOption{ddMMMMyyyy}{%
\renewcommand{\DatestampYMD}[3]{\mbox{#3 \LongMonth{#2} #1}}
\renewcommand{\DatestampYM}[2]{\mbox{\LongMonth{#2} #1}}
\renewcommand{\DatestampY}[1]{#1}}

% Datestamp format: {yyyy}{MM}{dd} ---> yyyy-MM-dd (e.g., 2008-01-15).
\DeclareOption{yyyyMMdd}{%
\renewcommand{\DatestampYMD}[3]{\mbox{#1-#2-#3}}
\renewcommand{\DatestampYM}[2]{\mbox{#1-#2}}
\renewcommand{\DatestampY}[1]{#1}}

% Datestamp format: {yyyy}{MM}{dd} ---> yyyy-MM (e.g., 2008-01).
\DeclareOption{yyyyMM}{%
\renewcommand{\DatestampYMD}[3]{\mbox{#1-#2}}
\renewcommand{\DatestampYM}[2]{\mbox{#1-#2}}
\renewcommand{\DatestampY}[1]{#1}}

% Datestamp format: {yyyy}{MM}{dd} ---> yyyy (e.g., 2008).
\DeclareOption{yyyy}{%
\renewcommand{\DatestampYMD}[3]{#1}
\renewcommand{\DatestampYM}[2]{#1}
\renewcommand{\DatestampY}[1]{#1}}

\ProcessOptions\relax

%%%%%%%%
% FONT %
%%%%%%%%

\setmainfont
[Path=./fonts/tinos/,
ItalicFont=Tinos-Italic,
BoldFont=Tinos-Bold,
BoldItalicFont=Tinos-BoldItalic]
{Tinos-Regular.ttf}

\newcommand{\UseSecondaryFont}{\fontspec
[Path=./fonts/gnu_free_font/,
ItalicFont=FreeSerifItalic,
BoldFont=FreeSerifBold,
BoldItalicFont=FreeSerifBoldItalic]
{FreeSerif.otf}}

\renewcommand{\sffamily}{\rmfamily}
\renewcommand{\ttfamily}{\rmfamily}
\renewcommand{\scshape}{\rmfamily}

\newcommand{\UseHeadingFont}{\normalfont}
\newcommand{\UseHeaderFooterFont}{\UseHeadingFont\fontsize{8.2pt}{9.5pt}\selectfont}
\newcommand{\UseNoteFont}{\UseHeadingFont\fontsize{8pt}{9.6pt}\selectfont}
\newcommand{\UseTitleFont}{\UseHeadingFont\fontsize{28pt}{33.6pt}\selectfont\bfseries}
\newcommand{\UseSubTitleFont}{\normalfont\fontsize{8.6pt}{10.3pt}\selectfont}
\newcommand{\UseSectionFont}{\UseHeadingFont\fontsize{9pt}{11pt}\selectfont\bfseries}
\newcommand{\UseSubSectionFont}{\UseHeadingFont\fontsize{8.6pt}{10.3pt}\selectfont\bfseries}
\newcommand{\UseDetailFont}{\normalfont\fontsize{8.6pt}{10.3pt}\selectfont}

\newcommand{\BulletSymbol}{{\normalfont\fontsize{6.5}{7.8}\selectfont\raisebox{0.17em}{\char"25A0}}}
\newcommand{\SubBulletSymbol}{{\normalfont\fontsize{6}{7.2}\selectfont\raisebox{0.17em}{\char"25CF}}}
\newcommand{\TildeSymbol}{{\normalfont\char"007E}}

%%%%%%%%%%%%%%%%%%%
% CUSTOM COMMANDS %
%%%%%%%%%%%%%%%%%%%

% \Title
% \SubTitle
% \Body
% \Gap
% \BigGap
% \Section
% \SubSection
% \Detail
% \BulletItem
% \SubBulletItem
% \Entry
% \Item
% \SubItem
% \NumberedItem
% \Hide

% Macro: Title (name).
\newcommand{\Title}[1]{%
\par\pdfbookmark[1]{#1}{#1}%
\begingroup\centering%
\UseTitleFont{#1}%
\par\endgroup\par}

% Macro: SubTitle (personal information below name).
\newenvironment{SubTitle}
{\par\vspace{0.25\baselineskip}%
\begingroup\centering%
\UseSubTitleFont}
{\par\endgroup\par}

% Macro: Body (rest of the document).
\newenvironment{Body}
{%
\par\vspace{1.5\baselineskip}%
\begin{list}
{}
{\setlength\leftmargin{1.2in}%
\setlength\rightmargin{0in}%
\setlength\labelwidth{1.2in}%
\setlength\labelsep{0in}%
\setlength\listparindent{0in}%
\setlength\itemindent{0in}%
\setlength\parskip{0in}%
\setlength\topsep{0in}%
\setlength\parsep{0in}%
\setlength\itemsep{0.90\baselineskip}%
\setlength\partopsep{0in}%
%\raggedright%
}%
}
{%
\end{list}%
\par%
}

% Macro: Gap (vertical gap between items in the same section).
\newcommand{\Gap}{\par\vspace{0.45mm}\par}

% Macro: BigGap (big vertical gap between items in the same section).
\newcommand{\BigGap}{\par\vspace{2.25mm}\par}

% Macro: Section (new section for Education, Research Experience, etc.).
\newcommand{\Section}[3]
{\par\pdfbookmark[2]{#2}{#3}%
\item[\smash{\parbox[t]{1.2in}{\UseSectionFont\raggedright\MakeUppercase{#1}}}]%
\par}

% Macro: SubSection.
\newcommand{\SubSection}[3]
{\par\pdfbookmark[3]{#2}{#3}%
{\UseSubSectionFont\raggedright\MakeUppercase{#1}}%
\par}

% Macro: Detail (text in smaller font under an item).
\newenvironment{Detail}
{\par\begingroup\UseDetailFont}
{\par\endgroup\par}

% Macro: BulletItem.
\newsavebox{\BulletItemIndentBox}
\newlength{\BulletItemIndentWidth}
%
\newcommand{\BulletItem}
{\par\vspace{0\baselineskip}%
\savebox{\BulletItemIndentBox}{\hspace{1.5mm}\BulletSymbol\hspace{1.5mm}}%
\settowidth{\BulletItemIndentWidth}{\usebox{\BulletItemIndentBox}}%
\parshape 2%
\labelwidth \linewidth%
\dimexpr\labelwidth+\BulletItemIndentWidth\relax \dimexpr\linewidth-\BulletItemIndentWidth\relax%
\usebox{\BulletItemIndentBox}%
\ignorespaces%
}

% Macro: SubBulletItem.
\newsavebox{\SubBulletItemIndentBox}
\newlength{\SubBulletItemIndentWidth}
%
\newcommand{\SubBulletItem}
{\par\vspace{0\baselineskip}%
\savebox{\SubBulletItemIndentBox}{\hspace{5.6mm}\SubBulletSymbol\hspace{1.5mm}}%
\settowidth{\SubBulletItemIndentWidth}{\usebox{\SubBulletItemIndentBox}}%
\parshape 2%
\labelwidth \linewidth%
\dimexpr\labelwidth+\SubBulletItemIndentWidth\relax \dimexpr\linewidth-\SubBulletItemIndentWidth\relax%
\usebox{\SubBulletItemIndentBox}%
\ignorespaces%
}

% Macro: Entry.
\newcommand{\Entry}
{\par\vspace{0\baselineskip}%
\parshape 1%
\labelwidth \linewidth%
\ignorespaces%
}

% Macro: SubEntry.
\newcommand{\SubEntry}
{\par\vspace{0\baselineskip}%
\parshape 1%
\labelwidth \linewidth%
\ignorespaces%
\filbreak%
}

% Macro: Item (aligns horizontally with BulletItem).
\newcommand{\Item}
{\par\vspace{0\baselineskip}%
\savebox{\BulletItemIndentBox}{\hphantom{\hspace{1.5mm}\BulletSymbol\hspace{1.5mm}}}%
\settowidth{\BulletItemIndentWidth}{\usebox{\BulletItemIndentBox}}%
\parshape 2%
\labelwidth \linewidth%
\dimexpr\labelwidth+\BulletItemIndentWidth\relax \dimexpr\linewidth-\BulletItemIndentWidth\relax%
\usebox{\BulletItemIndentBox}%
\ignorespaces%
}

% Macro: SubItem (aligns horizontally with SubBulletItem).
\newcommand{\SubItem}
{\par\vspace{0\baselineskip}%
\savebox{\SubBulletItemIndentBox}{\hphantom{\hspace{5.6mm}\SubBulletSymbol\hspace{1.5mm}}}%
\settowidth{\SubBulletItemIndentWidth}{\usebox{\SubBulletItemIndentBox}}%
\parshape 2%
\labelwidth \linewidth%
\dimexpr\labelwidth+\SubBulletItemIndentWidth\relax \dimexpr\linewidth-\SubBulletItemIndentWidth\relax%
\usebox{\SubBulletItemIndentBox}%
\ignorespaces%
}

% Macro: NumberedItem.
\newcommand{\MaxNumberedItem}{[88]}
%
\newsavebox{\MaxNumberedItemIndentBox}
\newlength{\MaxNumberedItemIndentWidth}
\newsavebox{\NumberedItemIndentBox}
\newlength{\NumberedItemIndentWidth}
%
\newcommand{\NumberedItem}[1]
{\par\vspace{0\baselineskip}%
\savebox{\MaxNumberedItemIndentBox}{\hspace{1.5mm}{\MaxNumberedItem}\hspace{2.5mm}}%
\settowidth{\MaxNumberedItemIndentWidth}{\usebox{\MaxNumberedItemIndentBox}}%
\savebox{\NumberedItemIndentBox}{\hspace{1.5mm}{#1}\hspace{2.5mm}}%
\settowidth{\NumberedItemIndentWidth}{\usebox{\NumberedItemIndentBox}}%
\parshape 2%
\labelwidth \linewidth%
\dimexpr\labelwidth+\MaxNumberedItemIndentWidth\relax \dimexpr\linewidth-\MaxNumberedItemIndentWidth\relax%
\hspace{\dimexpr\MaxNumberedItemIndentWidth-\NumberedItemIndentWidth\relax}%
\usebox{\NumberedItemIndentBox}%
\ignorespaces%
}

% Macro: Hide.
\newcommand{\Hide}[1]{}
